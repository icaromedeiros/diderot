\documentclass{report}
\usepackage{graphicx}
\usepackage[utf8]{inputenc}
\usepackage[toc,page]{appendix}

\linespread{1.5}
\begin{document}

\title{Marvin - A Test Driven Development Tool for developing RDF/OWL ontologies}
\author{
Ícaro Medeiros\\
Matrícula 1212403
\\ \\ \\ \\ \\ \\ \\ \\
Pontifical Catholic University of Rio de Janeiro \\
Doctorate in Computer Science \\
INF 2102 - Programming Final Project (Projeto final de Programação) \\
Coordinator: Prof. Arndt Von Staa \\
Advisor: Prof. Daniel Schwabe
}

\maketitle
\tableofcontents

\chapter{Introduction}

Test Driven Development (TDD) techniques have been successfully used in
software engineering \cite{beck03}, providing an iterative, agile and incremental way of developing
code. In this approach, programmers start by writing tests for their code, then programming the code
so the tests do not fail. Then code can be enhanced and checked against tests again, completing a desirable
test-write-refactor cycle.

TDD is a safe methodology to develop programs, where we can be sure that changes do not
introduce bugs or unexpected behavior by running our automated tests for our code. Moreover,
by doing tests first, we can check our architecture decisions simply by being clients of our own code
in our tests. Finally (?), the tests can be seen as a documentation about the code.

Almost all modern programming languages have unit test frameworks that enable to run
automated tests for your code and engage into a TDD process for software, such as the xUnit
framework \cite{beck03}.

In a TDD approach you can start with simple software units, in a bottom-up way,
rather than over modeling all your system in advance, the so called Big Design Up Front (BDUF).
This leads to code that can rapidly be released, in short cycles, enabling early releases of
shipable software for the clients to validate and provide feedback, a premise in Agile
Software Process Methodologies \cite{beck01, beck04, martin03}.

Ontology engineering could benefit from a TDD process like this, using small iterations
to ensure a good quality ontology, consistent, well tested and ready for evolution.

\section{Test Driven Development for Ontologies}

During the ontology development authors might not be aware what kinds of inferences their ontologies are
providing, for example, if a new rule introduced unexpected inference. Checking the ontology for
self-consistency is also a demanding task.

As stated in \cite{vrandevcic06}, the idea of design by contract, well known in Software Engineering can
be used to Ontology Engineering as follows: we could declare what statements should and
should not derive from an ontology being developed. Thus we can be sure if expected inferences are
derived and be aware of unexpected inferred conclusions.

Moreover, competency questions, as defined by some methodologies for Ontology Engineering like Methontology \cite{lopez99},
describe what kind of knowledge the resulting ontology is supposed to answer. These questions can always be formalized in
a query language (like SPARQL). By formalizing the queries and the expected answers, a system can automatically checks if
the ontology meets the requirements stated with the competency questions.

% example

To answer these needs in this report is proposed a novel Test Driven Development tool for building RDF/OWL ontologies, called
Marvin. This report is organized as follows: the Chapter \ref{system} describes the requirements, use cases, architecture and
how Marvin was built. In Chapter \ref{manual} it is described how to use Marvin in all its use cases, including installation.
Finally, in Chapter \ref{conclusion} we present concluding remarks and future goals for the project.

\chapter{System description}
\label{system}

This chapter will describe the goals defined for Marvin, a tool for Test Driven Development for building RDF/OWL ontologies
and how the system was built, including use cases, architecture and the engineering process.

\section{Goals}
\label{goals}

The main goals of Marvin is to provide a simple interface to test small portions of ontologies regarding:

\begin{itemize}
    \item Expected inferences. The system must check if expected inferences are satisfied by the ontology
        given as input.
    \item Check for unexpected inferences. The system must guarantee that no unexpected inferences are
        derived from the target ontology.
    \item Self consistency check. The system must check if rules stated in the ontology do not
        contradict each other thus leading to guaranteed ontology consistency.
    \item Answering competency questions. The system must check if answers to competency questions are
        the same as users expect, given the target ontologies and the knowledge base which uses the ontology.
\end{itemize}

\section{Requirements}

The requirements of Marvin follows:
\begin{itemize}
    \item To provide facilities for testing ontologies regarding the goals mentioned in Section \ref{goals}.
    \item To provide a simple Domain Specific Language for writting tests for ontologies. The API of the tool must be simple
        to use and extend to cover different test scenarios.
    \item To be written in Python. As the main language used in Globo.com having a tool written in Python is desirable.
        Moreover, the possibility to reuse tools such as RDFlib (for dealing with RDF data), the inference library FuXi
        and the Python unit testing framework lead to the requirement of using Python.
        Finally, Python is a multi-plataform language so the tool is available for all platforms that Python supports.
\end{itemize}

\subsection{Use cases}

\section{Architecture}

\section{Engineering process}

TDD

Version control: git, github

\subsection{Tests}

\chapter{User manual}
\label{manual}

\section{Installing}


\section{Documentation}

\section{Use examples}

\chapter{Conclusion}
\label{conclusion}

\section{Future Work}

\begin{appendices}
\chapter{CD content}

The CD ...

\end{appendices}

\bibliographystyle{plain}
\bibliography{refs}

\end{document}
